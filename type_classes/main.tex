\documentclass[11pt]{article}

\usepackage{sectsty}
\usepackage{graphicx}

\usepackage{listings}
\usepackage{xcolor}

\definecolor{dkgreen}{rgb}{0,0.6,0}
\definecolor{gray}{rgb}{0.5,0.5,0.5}
\definecolor{mauve}{rgb}{0.58,0,0.82}

\renewcommand{\familydefault}{\sfdefault}
\newcommand\func[1]{\textcolor{blue}{\texttt{#1}}}

\lstdefinestyle{myScalastyle}{
  frame=tb,
  language=scala,
  aboveskip=3mm,
  belowskip=3mm,
  showstringspaces=false,
  columns=flexible,
  basicstyle={\small\ttfamily},
  numbers=none,
  numberstyle=\tiny\color{gray},
  keywordstyle=\color{blue},
  commentstyle=\color{dkgreen},
  stringstyle=\color{mauve},
  frame=single,
  breaklines=true,
  breakatwhitespace=true,
  tabsize=3,
}
% Margins
\topmargin=-0.45in
\evensidemargin=0in
\oddsidemargin=0in
\textwidth=6.5in
\textheight=9.0in
\headsep=0.25in

\title{Scala type classes (Cats)}
\author{ Ramy Tanios }
\date{\today}

\begin{document}
\maketitle	
\pagebreak


\section{Cats type classes}
Sample implementations are presented below for the most important type classes supported by scala and Cats library.

\subsection{Semigroup}
A \textbf{Semigroup} is a type class that provides the \func{combine} method.
The following is the trait implementation of a \textbf{Semigroup}: 
\begin{lstlisting}[style=myScalastyle]
trait Semigroup[F]{ 
  def combine(l: F, r: F): F
}
\end{lstlisting}
The \func{combine} method satisfies the associativity propery $f(f(x,y), z) = f(x, f(y,z))$. 

\subsection{Monoid}
A \textbf{Monoid} is a type class that \textcolor{orange}{\textbf{extends}} a \textbf{Semigroup}, providing in addition to the 
\func{map} method, a \func{pure} or \func{unit} method. The following is a trait implementation of a \textbf{Monoid}: 
\begin{lstlisting}[style=myScalaStyle]
trait Monoid[F]{ 
  def unit: F 
  def combine(l: F, r: F): F
}
\end{lstlisting}
The former methods should follow the follow properties: 
\begin{itemize}
  \item $f(f(x,y), z) = f(x, f(y,z))$ (associativity). 
  \item $f(x, 0) = f(0, x) = x$. 
\end{itemize}



\subsection{Functor}
A \textbf{Functor} is a type class that provides the \func{map} method. 
The \func{map} serves as an \textcolor{magenta}{\textbf{ETW}} pattern: Extract, Transform and Wrap. The following 
is a trait implementation of a \textbf{Functor}: 
\begin{lstlisting}[style=myScalastyle]
trait Functor[F[_]]{ 
  def map[A, B](ma: F[A])(f: A => B): F[B]
}
\end{lstlisting}
The \func{map} method should have the following properties: 
\begin{itemize}
  \item $map(x)(identity) = x$
\end{itemize}
\subsection{Monad}
A \textbf{Monad} is a type class that provides the \func{pure}, \func{map} and \func{flatMap} methods. 
A monad \textcolor{orange}{\textbf{extends}} a \textbf{Functor}. The following is a trait implementation 
of a \textbf{Monad}: 
\begin{lstlisting}[style=myScalastyle]
trait Monad[F[_]]{
  def pure[A](a: A): F[A]
  def map[A, B](ma: F[A])(f: A => B): F[B]
  def flatMap[A, B](ma: F[A])(f: A => F[B]): F[B]
}
\end{lstlisting}

\subsection{Semigroupal}
A \textbf{Semigroupal} is a type class that provides the \func{product} method. The following is a trait implementation 
of a \textbf{Semigroupal}: 
\begin{lstlisting}[style=myScalastyle]
trait Semigroupal[F[_]]{ 
  def product[A, B](ma: F[A], mb: F[B]): F[(A,B)]
}
\end{lstlisting}
Note that a \textbf{Monad} \textcolor{orange}{\textbf{extends}} a \textbf{Semigroupal} as the \func{product} can be implemented 
using the \func{map} and \func{flatMap} methods as follows:
\begin{lstlisting}[style=myScalastyle]
def product[A, B](ma: F[A], mb: F[B]): F[(A,B)] = 
  ma.flatMap(a => mb.map(b => (a, b)))
}

def product[A, B](ma: F[A], mb: F[B]): F[(A,B)] = // for comprehension
  for {
    a <- ma 
    b <- mb 
  } yield (a,b)
\end{lstlisting}

\subsection{Applicative}
An \textbf{Applicative} is a type class that implements the \func{pure} and \func{map} methods.
Note that an \textbf{Applicative} \textcolor{orange}{\textbf{extends}} a \textbf{Functor}.
\begin{lstlisting}[style=myScalastyle]
trait Applicative[F[_]]{
  def pure[A](a: A): F[A]
  def map[A, B](ma: F[A], mb: F[B])(f: A => B): F[B]
  def ap[A, B](ma: F[A])(f: F[A => B]): F[B] // provided by scala
}
\end{lstlisting}
Note that an \textbf{Applicative} extends a \textbf{Semigroupal} as the former can implement 
the \func{product} through the \func{ap} method as follows: 
\begin{lstlisting}[style=myScalastyle]
def product[A, B](ma: F[A], mb: F[B]): F[(A,B)] = 
  ap(ma)(mb.map(b => ((a: A) => b)))
\end{lstlisting}



\end{document}
